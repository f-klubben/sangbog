\begin{sang}{En meget nostalgisk vise}{Melodi - Jeg plukker fløjlsgræs \ldots}
\spal2
\begin{vers}
Se, hist i staden i midnatstimen
en stud.scientum i kam'ret sad
Mens vinved blegned af frostens rimen
han lukked' bogen, ej mer han gad
{\sl Thi han ku' Rudin og analysen
og gamle Kreisky, den ingeniør
Men pædagoger gav fyren gysen
Thi de har blå ble og andefø'r}
\end{vers}
\begin{vers}
På basis lærte han byplanlægning
og skrev et værk på halvtredje ton
Her så han kun på bevis og sætning
og flytted' rundt på et epsilon
{\bf Han læste Rudin og analysen,
han læste Kreisky, den ingeniør
I kaffestuen så han med gysen
en pæd. med blå ble og andefø'r }
\end{vers}
\begin{vers}
På Mat 3 læste han Mellin-Olsen
og ham der Marx med sin merværdi
Fra lektor pæd.psyk. lød megen rosen
når gruppen opflammet stemte i:
{\sf Skråt op med Rudin og analysen !!
Til H. med Kreisky, den ingeniør !!
Vi pædagoger går uden gysen
med farvet ble og med andefø'r }
\end{vers}
\vbox{}\vfill
\begin{vers}
Som bifag valgte han EDB`en
med parserstakke og LIFO-list
og han blev fænget, han gik skam te`en
men måtte sande at han til sidst
{\sc ej mer' ku' Rudin og analysen
ej længer Kreisky, den ingeniør
men dataloger kan uden gysen
gå rundt i shorts og i andefø'r}
\end{vers}
\begin{vers}
På Mat 6 sku' han speciale skrive,
ideer havde han mange af
Men en af disse ku' ej det blive
Thi studienævnet har sagt, man ska' \vspace{1mm}
{\tt \small Ta' lidt fra Rudin, lidt analyse
og lidt fra Kreisky, den ingeniør
Men pædagoger vil ud vi fryse
Thi de har blå ble og andefø'r }
\end{vers}
\begin{vers}
Oh, stolte broder, den dag vil komme
da cand.scient.'er vi kaldes må
Du som adjunkt vil på prøve komme
og du vil følgende skudsmål få:
{\em Han ka' blot Rudin og analyse
og gamle Kreisky, den ingeniør
Han mat'matikken kan ej belyse
skønt han har tillagt sig andefø'r }
\end{vers}
\laps
\end{sang}