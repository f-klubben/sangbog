\nysang{Vendelboens festsang}{Melodi: Jyden han er stærk og sej}
\spal2
\begin{vers}
Wi er bøjen te fest i da
det er noe vi lier ... ja
men der er no osse det,
som ma wå noe kon betinkle we.
\end{vers}
\begin{vers}
Hwem mon vi ska ha te bur?
Wi vil gotnok helst ha muer.
såen en fremme kuen, ja hee
wil wi vær grow møj betinkle we.
\end{vers}
\begin{vers}
Wi ska se å konverser
damen, mens wi seer å  æer,
å wo snak, ka dej slog te?
Det er wi osse betinkle we.
\end{vers}
\begin{vers}
Hwa mon få no føe wi foer?
Skørrehat og tosseloer?
Ja, wi væ, de hør sæ te,
det ka wi got blyew betinkle we.
\end{vers}
\begin{vers}
Sån no fornem føe, we ma
da wes heller nægt å  ta
Nej, en bøf, -kartøfler te -
wil wi vær knap så betinkle we.
\end{vers}
\begin{vers}
Ka wi tåel det føe, we foer?
Hwa mæ hee wi for te bur?
Ska ma røeg, h{\o}r de sæ te?
Det er wi osse betinkle we.
\end{vers}
\vbox{}\vfill
\begin{vers}
Gawwen war jo nok wal dyer,
det vil de her kwejfolk styer,
men tho hwa, en småle entre -
det ska man entj vær betinkle we.
\end{vers}
\begin{vers}
Kwejfolk ku no dels wæ råer,
hwes de ku forsto å spåer,
ål wo pæng wo nok sat te,
det er wi grow møj betinkle we.
\end{vers}
\begin{vers}
Wi er jyder, wi ser "Nej".
Wi er nemle stærk å sæj,
men der er det sjøew we det:
kwonnern er wi betinkle we.
\end{vers}
\begin{vers}
Wi er kommen got i klæem,
wi sku nok vær bløwwen hjæm
we wo siel - men intj for det -
det er wi osse betinkle we
\end{vers}
\begin{vers}
For te fest det wil wi jaen,
wi ska nok bego wo pæent,
men de ajjer folk - ja si
dæm er wi møj mier betinkle we !!!
\end{vers}
\laps
\end{sang}