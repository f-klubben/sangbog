\begin{sang}{Elskabssang}{Melodi: I en kælder sort som kul}
\spal2
\begin{vers}
elskabsvenner, oleklart,
tadig mukkest unger.
tærke temmer ikkert nart
angens trofer lunger.
ung om olsort, ung om tær,
ung opran, ung olo kær,
ung om agnets kjalde, 
å kal temmer kralde.
\end{vers}
\begin{vers}
elskabsvenner amlet tå
tedets tolte ønner
tadig ine ``kæve lå''
ynger, veder, tønner.
pisebordet magfuldt tår
tore napseglas ødt pår,
mil om olen kinner, 
orte kyer vinder.
\end{vers}
\begin{vers}
enere å kaffes ``kænk''
omme veder varligt
øde må amt tore tænk,
om ig ømmer narlig.
lavecognac, tjerner-prit
iger omme mager kidt,
odavand ifoner 
prøjter, juser kummer.
\end{vers}
\begin{vers}
temningen tår tadig temt
tundom ulten lider,
trømtorsk tærke auce lemt
ærligt vælget vider.
pis om ultende oldat,
lig ur ennep om alat.
ulefadets ager  
elskabsvenner mager.
\end{vers}
\begin{vers}
ommetider elskabsleg
indet ammentrækker,
olde, vire, lemme lag
nart eks anser vækker.
pis å ure pegesild,
ynes triben ærlig nild,
å kal jældent kade 
tråle-tyrtebade.
narlig iger angen lut
topper krigen, krålen.
elskabsvenner end alut,
kænker, ynger, kråler:
yngekunsten tolt kal tå,
ene lægter lag kal lå.
ving å valedrikken, 
kål å iger kikken.
\end{vers}
\laps
\end{sang}