\begin{sang}{Glædelig Jul}{Melodi:Solskin ombord}
\begin{vers}
Vi nærmer os alle i denne tid, hvor hver mand
går rundt i butikker og lader sit
blik løbe hen over alt, der kan fås.
Han ser på butikspigens dejlige 
udvalg af gaver i småt og stort,
men syn's, at det alt sammen er noget
dyrt, når man ikke har mere i løn,
og konen naturligvis venter en
glædelig jul, jul som i gamle dage,
flettede hjerter i guld og sølv.
Konen vil ha' sig en glædelig jul.
\end{vers}
\begin{vers}
Man samles i kreds om det festlige bord,
og far stikker kniven i brystet på
gåsen, mens mor med taknemmeligt sind
omsider får lov til at slippe en
stund for det hæslige mas og besvær,
der slider på en, der må leve som
husmor, når hjemmet skal passes med flid.
Var hun ikke til, fik de aldrig en
glædelig jul, jul som i gamle dage,
flettede hjerter i guld og sølv,
så fik de aldrig en glædelig jul.
\end{vers}
\begin{vers}
Det tindrer i øjet på pige og dreng,
i aften skal de med hinanden i
hånden gå rundt om det pyntede træ,
og nissen, der kommer, er cand.scient med
vatskæg og gaver til alle og en,
han plejer ved træet at løfte sit
bæger, når der bli'r sagt: Værs'go og skyl!
og så skal de endelig ha' sig en
glædelig jul, jul som i gamle dage,
flettede hjerter i guld og sølv,
så ska' de ha' sig, ja så ska' de ha' sig,
ja, så ska' de ha' sig en glædelig jul.
\end{vers}
\laps
\end{sang}